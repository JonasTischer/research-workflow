\chapter{Results}
\label{ch:results}

\section{Main Results}

% TODO: Present your main findings
Table~\ref{tab:main-results} presents our main experimental results.

\begin{table}[h]
\centering
\begin{tabular}{lccc}
\hline
\textbf{Model} & \textbf{Accuracy} & \textbf{F1} & \textbf{BLEU} \\
\hline
Baseline & 78.3 & 0.76 & 24.1 \\
Our Method & \textbf{85.7} & \textbf{0.84} & \textbf{31.2} \\
\hline
\end{tabular}
\caption{Main experimental results on the test set.}
\label{tab:main-results}
\end{table}

\section{Comparison with Baselines}

Our approach outperforms existing methods by a significant margin.
Compared to the baseline transformer \citep{vaswani2017attention},
we achieve a 7.1 BLEU improvement.

% VERIFY: This claim should match the table data above

\section{Ablation Study}

To understand the contribution of each component...

\begin{table}[h]
\centering
\begin{tabular}{lc}
\hline
\textbf{Configuration} & \textbf{Accuracy} \\
\hline
Full Model & 85.7 \\
- Component A & 82.1 \\
- Component B & 83.4 \\
- Component C & 81.9 \\
\hline
\end{tabular}
\caption{Ablation study results.}
\label{tab:ablation}
\end{table}

\section{Qualitative Analysis}

% TODO: Add qualitative examples
Figure~\ref{fig:example} shows an example of...

\section{Discussion}

Our results demonstrate that...

\subsection{Limitations}

This study has several limitations:
\begin{itemize}
    \item Limited dataset size
    \item Domain-specific training data
    \item Computational constraints
\end{itemize}
